\RequirePackage{amsmath}
\documentclass{article}
\usepackage{amssymb}
\usepackage{tikz}
\tikzstyle{arrow}=[-angle 45]
\usepackage{tikz-cd}
\usepackage{wrapfig}
\usepackage[final]{listings}
\usepackage{mathptmx}
\usepackage{stmaryrd,mathtools}
\let\vec\relax
\usepackage{MnSymbol}
%\usepackage{natbib}
\usepackage{hyperref}
\usepackage{url}
\usepackage{comment}
\usepackage{listings}
\usepackage{fancyvrb}
\usepackage{xspace}
\usepackage[show]{ed}

% \pagestyle{headings}


\begin{document}
When constructing the MathScheme library, here are some points I thought are worth sharing. 
\begin{itemize}
	\item constructing the diagram in the paper for building monoids, is not possible. It is possible to get everyone of these theories, but not construct all the arrows only by combine.
	Here are some examples, of why this is a problem: 
	\begin{itemize}
		\item When constructing the theory of \verb|AdditivePointedMagma|, one would want to see all the arrows that make sense
		\begin{itemize}
			\item An extension from AdditiveMagma 
			\item An extension from Pointed0 
			\item A rename of PointedMagma 
		\end{itemize}
	    If we are to do this in one declaration, we either get the first two\footnote{combine AdditiMagma {} and Pointed0 {} over Carrier} or the last one\footnote{rename PointedMagma {op to + ; e to 0}}. In order to get all these relations, one has to create two more theories smaller theories, which is more relistically presented in figure \ref{fig:APM}. 
	    \begin{Verbatim}[fontsize=\small]
Pointed0Magma = combine Pointed0 {} PointedMagma zero over Pointed
PointedPlusMagma = combine AdditiveMagma {} PointedMagma plus over Magma
AdditivePointedMagma = combine Pointed0Magma plus PointedPlusMagma zero over PointedMagma 	    
	    \end{Verbatim}
	\item A more interesting problem can be seen when trying to get all the arrows to construct \verb|AddUnital|. In this case we want the following arrows
	\begin{itemize}
		\item A rename arrow from \verb|Unital| to \verb|AddUnital|
		\item An extension arrow from \verb|LeftUnital+| to \verb|Unital| 
		\item An extension arrow from \verb|RightUnital+| to \verb|Unital| 
	\end{itemize}
	The problem here is more serious because the following two combine statements yields to the same result. Performing the combine twice will create two similar theories with different names, and the arrows will be distributed among those theories.  
	\begin{Verbatim}[fontsize=\small]
AddUnital  = combine AddRightUnital {} Unital plus-zero over RightUnital
AddUnital' = combine AddLeftUnital {} Unital plus-zero over LeftUnital 
	\end{Verbatim}
	\item We'll see the same problem when attempting to define \verb|AdditiveMonoid|
	\begin{itemize}
		\item In case we combine \verb|Monoid| with \verb|AddUnital|, we lose connection with \verb|AddSemigroup|
		\begin{Verbatim}[fontsize=\small]
AddMonoid = combine Monoid plus-zero AddUnital {} over Unital
		\end{Verbatim}
		\item In case we combine \verb|Monoid| with \verb|AddSemigroup|, we lose the connection with\verb|AddUnital| 
		\begin{Verbatim}[fontsize=\small]
AddMonoid' = combine Monoid plus-zero AddSemigroup {} over Unital
	    \end{Verbatim}	
		\item In case we combine \verb|AddSemigroup| and \verb|AddUnital|, we lose connection with \verb|Magma|! 
	    \begin{Verbatim}[fontsize=\small]
AddMonoid'' = combine AddUnital {} AddSemigroup {} over AddMagma
       \end{Verbatim}		
	\end{itemize}
	\item Another example of a theory that may be defined in different ways, leading to different arrows is \verb|InvolutivePointedMagmaSig|. The two definitions are 
	\begin{verbatim}[fontsize=\small]
InvolutivePointedMagmaSig  = combine UnaryOperation {} PointedMagma {} over Carrier 
InvolutivePointedMagmaSig' = combine PointedUnarySystem {} Magma {} over Carrier 
	\end{verbatim}
	\item In some cases, we need to transport arrows, such that a theory becomes connected to another 
	\end{itemize} 
\end{itemize}
The code for defining \verb|AdditiveMonoid|: 
	\begin{Verbatim}[fontsize=\small]
Map plus = {op to +}
Map zero = {e to 0}
Map plus-zero = {op to + ; e to 0}
Theory Empty = {} 
Carrier = extend Empty {A : Set}
Pointed = extend Carrier {e : A}
Pointed0 = rename Pointed zero 

Magma = extend Carrier {op : A -> A -> A}
AdditiveMagma = rename Magma plus

PointedMagma = combine Pointed {} Magma {} over Carrier
Pointed0Magma = combine Pointed0 {} PointedMagma zero over Pointed
PointedPlusMagma = combine AdditiveMagma {} PointedMagma plus over Magma
AdditivePointedMagma = combine Pointed0Magma plus PointedPlusMagma zero over PointedMagma

LeftUnital = extend PointedMagma { lunit_e : {x : A} -> op e x == x }
RightUnital = extend PointedMagma { runit_e : {x : A} -> op x e == x }
Unital = combine LeftUnital {} RightUnital {} over PointedMagma
AddLeftUnital = combine AdditivePointedMagma {} LeftUnital plus-zero over PointedMagma 
AddRightUnital = combine AdditivePointedMagma {} RightUnital plus-zero over PointedMagma
AddUnital = combine AddRightUnital {} Unital plus-zero over RightUnital

Semigroup = extend Magma {associative_op : {x y z : A} -> op (op x y) z == op x (op y z) }
AdditiveSemigroup = combine AdditiveMagma {} Semigroup plus over Magma

Monoid = combine Unital {} Semigroup {} over Magma 
AddMonoid = combine Monoid plus-zero AddUnital {} over Unital
	\end{Verbatim}
\begin{figure}
	\begin{tikzcd}[row sep=0.9em, column sep=0.3em]
	&& & & \verb|Pointed0| \arrow[dd,hook]  \\
	\verb|Carrier| \arrow[dd,hook] \arrow[rr,hook] & & \verb|Pointed| \arrow[urr,mapsto] & & \\
	& \verb|Magma+| \arrow[rr,hook]& & \verb|PointedPlusMagma|\arrow[dr,mapsto]& \verb|Pointed0Magna| \arrow[d,mapsto]\\
		\verb|Magma| \arrow[ur,mapsto]  & &
	\verb|PointedMagma| \arrow[ur,mapsto] \arrow[urr,mapsto]\arrow[from=uu, crossing over]\arrow[from=ll,hook, crossing over] && \verb|PointedMagma+|
	\end{tikzcd}
\caption{How constructing AdditivePointedMagma really looks like}
\label{fig:APM}
\end{figure}

\begin{figure}
	\begin{tikzcd}[row sep=0.9em, column sep=0.3em]
		&& & \verb|Pointed0| \arrow[dd,hook] & \\
		\verb|Carrier| \arrow[dd,hook] \arrow[rr,hook] & & \verb|Pointed| \arrow[ur,mapsto] & & \\
		& \verb|Magma+| \arrow[rr,hook] \arrow[dd,hook]& & \verb|PointedMagma+|
		\arrow[rr,hook]
		\arrow[dd,hook]& &
		\verb|RightUnital+|  \arrow[dd,hook]\\
		\verb|Magma| \arrow[ur,mapsto] \arrow[dd,hook]  & &
		\verb|PointedMagma| \arrow[dd,hook] \arrow[ur,mapsto] \arrow[from=uu, crossing over]
		\arrow[rr,hook,crossing over] \arrow[from=ll,hook, crossing over]
		& & \verb|RightUnital| \arrow[ur,mapsto] \\
		& \verb|Semigroup+| \arrow[ddd,hook] & &\verb|LeftUnital+| \arrow[rr,hook] &  &
		\verb|Unital+| \arrow[dddllll,hook] \\
		\verb|Semigroup|  \arrow[ddd,hook] \arrow[ur,mapsto]&  & \verb|LeftUnital|
		\arrow[ur,mapsto]
		\arrow[rr,hook,crossing over] &  & \verb|Unital| \arrow[dddllll,hook,crossing
		over]\arrow[ur,mapsto]
		\arrow[from=uu,hook,crossing over]& \\
		&&&& \\
		&  \verb|Monoid+| &  &&\\
		\verb|Monoid|  \arrow[ur,mapsto] && &&
	\end{tikzcd}
	\caption{Structure of the algebraic hierarchy up to Monoids}
	\label{fig:cube2_monoid}
\end{figure}

\end{document}
